% !TeX root = ./main.tex

% TEMPLATE thesis UniPD

\documentclass[a4paper,12pt]{memoir} % memoir is book with abstract support


%%%%%%%%%%%%%%%%%%%%%%%%%%%%%%%%PACKAGES%%%%%%%%%%%%%%%%%%
\usepackage[font=small,labelfont=bf]{caption} %Makes image caption small



\usepackage[a-1b]{pdfx} % Generate a PDF/A

\pdfpagewidth\paperwidth
\pdfpageheight\paperheight


\usepackage[italian, english]{babel} %italian
\usepackage[utf8]{inputenc}
\usepackage[T1]{fontenc}

\usepackage{lmodern,amsmath,amssymb,graphicx}
%Subfloat figure
\usepackage{subfig}
\usepackage{wrapfig}
\usepackage{adjustbox}
\usepackage{graphicx} 

%inline code
\usepackage{listings}

\lstset{
	basicstyle=\ttfamily,
	columns=flexible,
}


%Wrapfigure
\usepackage{wrapfig}
\usepackage{float}

%Tables
\usepackage{multirow}

% Useful for SI units:
\usepackage[per-mode=fraction,binary-units]{siunitx}

% Useful for bibliography:
\usepackage[
%backend=biber, 
natbib=true,
style=numeric,
sorting=none
]{biblatex}
\addbibresource{Bibliography/bibl.bib}

%doppio titolo 
\usepackage{titling}
\newcommand{\subtitle}[1]{%
	\posttitle{%
		\par\end{center}
	\begin{center}\large#1\end{center}
	\vskip0.5em}%
}

% If you want to use subsections:
\setsecnumdepth{subsection}

\aliaspagestyle{chapter}{empty} % No page number on new chapter page

% Small margins:
\setulmarginsandblock{3cm}{*}{*}
\setlrmarginsandblock{2cm}{*}{*}
\checkandfixthelayout

\title{\textbf{Inverse Beta Decay events selection in JUNO using Machine Learning algorithms} \\}
\subtitle{Selezione di eventi di decadimento beta inverso in JUNO utilizzando algoritmi di apprendimento automatico}
\author{Fabio Cufino}

% METADATA for PDF
\begin{filecontents*}{\jobname.xmpdata}
\Title{Thesis title}
\Author{Fabio Cufino}
\Language{en-US} %{en-US}
\Date{YYYY-MM-DD}
\Keywords{keyword1\sep keyqord2\sep keyword3}
\end{filecontents*}

\hypersetup{hidelinks} % Hide link color boxes on pdf


\begin{document}

%%%%%%%%%%%%%%%%%%%%%%%%%%%%%%%%%%%%%% COVER %%%%%%%%%%%%%%%%%%%%%%%%%%%%%%%%%%%%%%
  \frontmatter
  \begin{titlingpage} % titlepage in book
    \vspace{5mm}
    \begin{figure}[ht]
      \centering
      \includegraphics[scale=.13]{Images/logo.png}
    \end{figure}
    \vspace{5mm}
    \begin{center}
      {{\Large{\textsc{\textbf{UNIVERSITÀ DEGLI STUDI DI PADOVA}}}}\\}
      \vspace{5mm}
      {\textbf{Dipartimento di Fisica e Astronomia}} \\ % Department of XXXXXXXXXXXX
      \vspace{5mm}
      {\textsc{\textbf{Corso di Laurea Triennale in Fisica}}}\\ % Master's Degree in
      \vspace{20mm}
      {\textsc{\textbf{Tesi di Laurea}}}\\ % Final Dissertation
      \vspace{25mm}
      \begin{Spacing}{2}
        {\Large \thetitle}
      \end{Spacing}
        \begin{Spacing}{1}{Selezione di eventi di decadimento beta inverso in JUNO utilizzando algoritmi di apprendimento automatico}
  \end{Spacing}
      \vspace{7mm}
    \end{center}

    \vspace{15mm}
    \begin{Spacing}{1.5}
      \begin{tabular}{p{0.4\textwidth} p{0.3\textwidth} p{0.3\textwidth}}
        Relatore && Laureando\\ % Supervisor && Candidate
        \textbf{Prof. Alberto Garfagnini} && \textbf{\theauthor}\\
        Correlatore && \textbf{}\\ % Co-supervisors
        \textbf{Dott. Andrea Serafini} && \\ % Dr.
      \end{tabular}
    \end{Spacing}
    \vspace{8 mm}
    \begin{center}
      \textbf{Anno Accademico 2022/2023} % Academic Year
          \vspace{5 mm}\\\textbf{}
    \end{center}
  \end{titlingpage}
  \clearpage{\pagestyle{empty}\cleardoublepage}

%%%%%%%%%%%%%%%%%%%%%%%%%%%%%%%%%%%%%% ABSTRACT %%%%%%%%%%%%%%%%%%%%%%%%%%%%%%%%%%%%%%
  \begin{abstract}
   The Jiangmen Underground Neutrino Observatory (JUNO) will be the largest liquid scintillator-based neutrino detectors in the World, for the next decade. Thanks to its large active mass (20 kt) and state of the art performances (3\% effective energy resolution at 1 MeV), it will be able to perform important measurements in neutrino physics. This study focuses on improving the event selection performance in JUNO by applying a machine learning techniques, specifically Boosted Decision Trees (BDT) and Neural Networks (NN), to discriminate between signal events, that are interactions of anti-neutrinos coming from the nearby nuclear power plants from uncorrelated background events.
   The BDT model demonstrated exceptional performance in event classification, achieving high precision and accuracy in distinguishing between signal events and background events. It achieved a remarkably low number of misclassifications, by surpassing the number of misclassified events by the Manual Cut, the current state-of-the-art for classification.
   The Neural Network model also exhibited strong performance, with a slightly higher number of misclassifications compared to BDT. However, it showcased a notable capability to accurately identify signal events while maintaining a relatively low number of misclassifications for background events. 

  \end{abstract}

  \begin{abstract}
   Il Jiangmen Underground Neutrino Observatory (JUNO) sarà il più grande rilevatore a scintillazione liquido di neutrini al mondo, per il prossimo decennio. Grazie alla sua grande massa attiva (20 kt) e alle prestazioni all'avanguardia (3\% di risoluzione energetica a 1 MeV), sarà in grado di effettuare importanti misurazioni nella fisica dei neutrini. Questo studio si concentra sul miglioramento delle prestazioni di selezione degli eventi in JUNO applicando tecniche di apprendimento automatico, nello specifico Boosted Decision Trees (BDT) e Neural Networks (NN), per discriminare tra eventi di segnale, che sono interazioni di anti-neutrini provenienti dalle vicine centrali nucleari, e eventi di fondo non correlati.
   Il modello BDT ha dimostrato prestazioni eccezionali nella classificazione degli eventi, ottenendo una alta precisione e accuratezza nel distinguere tra eventi di segnale e eventi di fondo. Ha raggiunto un numero notevolmente basso di errori di classificazione, superando il numero di eventi erroneamente classificati dal Manual Cut, l'attuale stato dell'arte per la classificazione.
   Il modello della NN ha mostrato prestazioni eccellenti, con un numero leggermente maggiore di errori di classificazione rispetto al BDT. Tuttavia, ha dimostrato una notevole capacità di identificare accuratamente gli eventi di segnale mantenendo un numero relativamente basso di errori di classificazione per gli eventi di fondo.
   \end{abstract}
   
  \newpage
%%%%%%%%%%%%%%%%%%%%%%%%%%%%%%%%%%%%%% TABLE OF CONTENTS %%%%%%%%%%%%%%%%%%%%%%%%%%%%%%%%%%%%%

  \tableofcontents
  \newpage
  %\listoffigures
  %\listoftables

%%%%%%%%%%%%%%%%%%%%%%%%%%%%%%%%%%%%%% BODY %%%%%%%%%%%%%%%%%%%%%%%%%%%%%%%%%%%%%%

  \mainmatter

  \chapter{Introduction}

The Jiangmen Underground Neutrino Observatory (JUNO) \cite{JUNOdet}, currently under construction in south China, is a large liquid scintillator neutrino detector. It is designed to detect electron antineutrino interactions produced by nearby Nuclear Power Plants (NPP) through the inverse beta decay reaction. The primary objective of this experiment is to determine the neutrino mass hierarchy, addressing the Neutrino Mass Ordering (NMO) problem.

The JUNO experiment, thanks to its excellent energy resolution and large fiducial volume, is expected to make significant contributions to this field.

\section{Neutrino Oscillations}
Neutrinos, originally assumed to be massless, have been experimentally proven to have non zero mass. This result is mainly due to the discovery of neutrino oscillations, a quantum mechanical phenomenon where a neutrino changes its flavor during propagation, providing strong evidence of their mass \cite{dim_osci_neut}.

Each known flavor eigenstate, $(\nu_e,\nu_{\mu},\nu_{\tau})$, linked to three respective charged leptons $(e,\mu,\tau)$  via the charged current interactions can be considered a complex combination of neutrino mass eigenstates as follow:

\begin{equation*}
	\left(\begin{array}{l}
		v_e \\
		v_\mu \\
		v_\tau
	\end{array}\right)=U_{\mathrm{PMNS}}\left(\begin{array}{l}
		v_1 \\
		v_2 \\
		v_3
	\end{array}\right)
\end{equation*}
in which $\nu_i$ are the three mass eigensates, with non-degenerate masses  $m_i$  $(i = 1,2,3)$.\\

The matrix $U_{\mathrm{PMNS}}$ \cite{matrix_1}, \cite{matrix_2}, \cite{matrix_3}, the so-called Pontecorvo-Maki-Nakagawa-Sakata (PMNS) matrix, is composed of three rotation matrices, $R_{23}$, $R_{13}$, and $R_{12}$, each corresponding to a different mixing angle, $\theta_{23}$, $\theta_{13}$, and $\theta_{12}$, respectively and a parameter $\delta_{CP}$ called the Dirac CP-violating phase. In addition, the Majorana $C P$ phases $\eta_i(i=1,2)$, are only physically possible if neutrinos are Majorana-type particles and do not participate in neutrino oscillations. Therefore, $U_{\mathrm{PMNS}}$ can be expressed as:

\begin{equation*} 
	\begin{split}
			U_{\text {PMNS }}&=\\
		=&\left(\begin{array}{ccc}
			1 & 0 & 0 \\
			0 & c_{23} & s_{23} \\
			0 & -s_{23} & c_{23}
		\end{array}\right) \left(\begin{array}{ccc}
			c_{13} & 0 & s_{13} \mathrm{e}^{-\mathrm{i} \delta_{C P}} \\
			0 & 1 & 0 \\
			-s_{13} \mathrm{e}^{\mathrm{i} \delta_{C P}} & 0 & c_{13}
		\end{array}\right) 
		\left(\begin{array}{ccc}
			c_{12} & s_{12} & 0 \\
			-s_{12} & c_{12} & 0 \\
			0 & 0 & 1
		\end{array}\right)\left(\begin{array}{ccc}
			\mathrm{e}^{\mathrm{i} \eta_1} & 0 & 0 \\
			0 & \mathrm{e}^{\mathrm{i} \eta_2} & 0 \\
			0 & 0 & 1
		\end{array}\right)
	\end{split}
\end{equation*}
where $s_{i j} \equiv \sin \theta_{i j}, c_{i j} \equiv \cos \theta_{i j}$.

The theoretical framework for neutrino oscillations involves the calculation of the oscillation probability as a function of the distance traveled by the neutrino, the neutrino mixing matrix, and the difference in squared masses between the three neutrino mass states, $\Delta m_{ij}^2 = m^2_i - m^2_j$ for $i,j = 1,2,3, i>j$. In the case of JUNO, two nuclear power reactors 53 $\unit{\kilo\meter}$ away from the detector, produce electron anti-neutrinos $\bar{\nu}_e$ with energy below 10 MeV, as principal sources of neutrinos for the experiment. Therefore, the survival probability $P\left(\bar{\nu}_e \rightarrow \bar{\nu}_e\right)$ of electron antineutrinos reads:

\begin{equation*}
	P\left(\bar{\nu}_e \rightarrow \bar{\nu}_e\right)=1-\sin ^2 2 \theta_{12} c_{13}^4 \sin ^2\left(\frac{\Delta m_{21}^2 L}{4 \mathcal{E}}\right)-\sin ^2 2 \theta_{13}\left[c_{12}^2 \sin ^2\left(\frac{\Delta m_{31}^2 L}{4 \mathcal{E}}\right)+s_{12}^2 \sin ^2\left(\frac{\Delta m_{32}^2 L}{4 \mathcal{E}}\right)\right]
\end{equation*}

where $\mathcal{E}$ is the neutrino energy, $L$ the travelled distance and $\Delta m_{i j}^2 \equiv m_i^2-m_j^2$. \\
Past experiments have already given estimates for $\Delta m_{21}^2,\left|\Delta m_{31}^2\right|$ and the three mixing angles.

Current values for the neutrino oscillation parameters are reported in Table \ref{tab:neutrino_params} \cite{pdg}. The last column denotes the relative uncertainties, expressed in percentage.


\begin{table}[h]
\centering
\small
\begin{tabular}{ccc}
\hline
\toprule
\textbf{Parameter} & \textbf{Value} & \textbf{Relative Uncertainty} \\ \midrule
$\Delta m^2_{32}$ (NO) & $(2.453 \pm 0.034) \times 10^{-3} \, \text{eV}^2$ & 1.4\% \\
$\Delta m^2_{32}$ (IO) & $-(2.546 \pm 0.037) \times 10^{-3} \, \text{eV}^2$ & 1.5\% \\
$\Delta m^2_{21}$ & $(7.53 \pm 0.18) \times 10^{-5} \, \text{eV}^2$ & 2.4\% \\
$\sin^2 \theta_{12}$ & $0.307 \pm 0.013$ & 4.2\% \\
$\sin^2 \theta_{13}$ & $0.0218 \pm 0.0007$ & 3.2\% \\ \bottomrule
\hline
\end{tabular}
\caption{Neutrino oscillation parameters within the reach of JUNO and their 1$\sigma$ uncertainties, as reported in PDG2020. \cite{pdg}}
\label{tab:neutrino_params}
\end{table}

JUNO's primary objective is to improve these measurements and to determine the sign of $\Delta m_{31}^2$, which will distinguish between two potential scenarios:
\begin{itemize}
	\item \textit{Normal Ordering (NO)}, where $\left|\Delta m_{31}^2\right|=\left|\Delta m_{32}^2\right|+\left|\Delta m_{21}^2\right|$ and the mass hierarchy is $m_1<m_2<m_3$,
	\item \textit{Inverted Ordering (IO)}, where $\left|\Delta m_{31}^2\right|=\left|\Delta m_{32}^2\right|-\left|\Delta m_{21}^2\right|$ and the mass hierarchy is $m_3<m_1<m_2$.
\end{itemize}
The sign of $\Delta m_{31}^2$ alters the curves of Figure \ref{fig:antineutinotoantineutrinoprobabilityplot}.



\begin{figure}[h]
	\begin{minipage}{.4\textwidth}
		\caption{JUNO's reactor antineutrino energy spectrum is shown with and without the effect of neutrino oscillation. The gray dashed curve includes only the term in the disappearance probability modulated by $sin^2(2\theta_{12})$, while the blue and red curves use the full oscillation probability for normal and inverted mass orderings. Spectral features driven by oscillation parameters are illustrated, highlighting the rich information available in JUNO's high-resolution measurement of the oscillated spectrum. \cite{Sub_osci}}
		\label{fig:antineutinotoantineutrinoprobabilityplot}
	\end{minipage}%
	\hspace{1cm}
	\begin{minipage}{.5\textwidth}
		\includegraphics[width=\linewidth]{Images/antineutino_to_antineutrino_probability_plot}
	\end{minipage}
\end{figure}

\newpage




\section{The JUNO detector}
The Jiangmen Underground Neutrino Observatory is currently being constructed beneath Dashi Hill in the town of Jinji, Southern China. Its placement 43 km southwest of Kaiping city was strategically chosen to significantly reduce background  from cosmic rays due to its underground location. JUNO is anticipated to detect antineutrinos, predominantly originating from the Taishan and Yangjiang nuclear power plants. NPPs are approximately 52.5 km away from the JUNO detector and will provide a combined nominal thermal power of 26.6 $GW_{th}$.\\

A schematic illustration of JUNO is presented in Fig.\ref{fig:junoschemeexperiment}.


\begin{figure}[h]
	\centering
	\includegraphics[width=0.6\linewidth]{Images/juno_scheme_experiment}
	\caption[JUNO scheme experiment]{Schematic view of the JUNO experiment, from \cite{JUNOdet}}
	\label{fig:junoschemeexperiment}
\end{figure}

The core of the JUNO detector, the Central Detector (CD), is complemented by a water Cherenkov detector and a Top Tracker (TT). Notably, the CD, designed as a compact, non-segmented detector, boasts an effective energy resolution of $\sigma_E/E =3\% / \sqrt{E (MeV)}$] \cite{JUNOdet}.\\

The CD houses 20 kton of liquid scintillator (LS) within a spherical acrylic vessel, submerged in a water pool. This pool, with a diameter of 43.5 m and a height of 44 m, not only serves as an effective buffer to shield the LS from the radioactive influence of the surrounding rock, but it acts as active scintillation medium. Equipped with PMTs for Cherenkov detection, the pool plays an instrumental role in the active shielding process.

The vessel is supported by a stainless steel (SS) structure through connecting bars. Additional CD PMTs are mounted on the inner surface of this structure, which also hosts compensation coils designed to mitigate the Earth's magnetic field and thereby minimize its impact on the photoelectron collection efficiency of the PMTs.

Above the water pool resides the Top Tracker, an assembly of a plastic scintillator array, meticulously arranged to measure muon tracks accurately. The CD is connected to the external environment through a chimney, which facilitates calibration operations. Located above this chimney is the Calibration House, equipped with special radioactivity shielding and a muon detector, playing a crucial role in the overall experimental setup.

JUNO experiment deploys a specialized compact detector named TAO. Situated approximately 30 meters from one of the Taishan reactors, TAO serves to measure the unoscillated reactor antineutrino spectrum shape precisely. The data collected by TAO is intended to provide a crucial data-driven input to refine the spectra from the other reactor cores. \\

\newpage

\section{JUNO signal and background}

\subsection{Signal}
The JUNO experiment primarily draws its sources from the Taishan and Yangjiang NPPs, which are made of two and six cores, respectively. In addition to these, the Daya Bay reactor complex contributes to the antineutrino flux. The reactor power and baselines from the Taishan, Yangjiang, and Daya Bay reactor cores are detailed in Table \ref{tab:IBD_reactor_source}.


\begin{table}[htp]
	\centering
        \small
	\begin{tabular}{ccc}
		\toprule
		Reactor & Power [$GW_{th}$] & Baseline [Km]   \\\midrule
		Taishan & 9.2 & 52.71 	\\
		Yangjiang & 17.4 & 52.46 	\\
		Daya Bay & 17.4 & 215 \\
		\bottomrule
	\end{tabular}
	\caption{Information on nuclear reactors}
	\label{tab:IBD_reactor_source}
\end{table}


JUNO employs a Liquid Scintillator primarily composed of Linear Alkyl-Benzene (LAB), known for its transparency, high flash points, robust light yield, and low chemical reactivity. The LS, with a density of 0.859 $g/mL$, is further enhanced with 3 $g/L$ of 2,5-diphenyloxazole (PPO) as the fluor, and 15 $mg/L$ of p-bis-(o-methylstyryl)-benzene (bis-MSB) as the wavelength shifter. \\

This process is initiated when an antineutrino interacts with a proton in the liquid scintillator, producing a positron and a neutron. It can be described by the following Inverse Beta Decay (IBD) reaction:

\begin{equation}
	\begin{aligned}
		\overline{\nu}_e + p &\rightarrow e^+ + n \\
	\end{aligned}
\end{equation}

IBD is characterized by a comparatively low threshold of 1.8 MeV, a substantial cross section, and it can be readily differentiated from the background due to its delayed $\gamma$ signature.\\

The positron, carrying the majority of the antineutrino's initial energy, deposits this energy in the scintillator through ionization. This energy deposition, coupled with the positron's subsequent annihilation typically into two 0.511 MeV photons, forms the \textbf{prompt signal}, characterized as follow: $e^{+} + e^{-} \rightarrow 2\gamma$.
The energy deposited by the positron directly correlates with the antineutrino energy, providing a precise measure critical for neutrino oscillation studies.\\

Following the prompt signal, the neutron is captured primarily on hydrogen (approximately 99$\%$ of the time) after an average delay of about 220 µs. This capture event releases a single 2.2 MeV photon, creating the \textbf{delayed signal}. Occasionally, the neutron is captured on carbon (around 1$\%$ of the time), resulting in a gamma-ray signal with a total energy of 4.9 MeV \cite{Sub_osci}. The processes are described as follows:

\begin{equation}
	 n + ^{1}\text{H} \rightarrow ^{2}\text{H}^{*} \rightarrow ^2\text{H} + \gamma
\end{equation}

\begin{equation}
n + ^{12}\text{C} \rightarrow ^{13}\text{C}^{*} \rightarrow ^{13}\text{C} + \gamma
\end{equation}

The scintillation light output originating from these events is detected by the photomultiplier tubes (PMTs), sensitive detectors that convert light into an electrical signal. They operate based on the photoelectric effect and subsequent electron multiplication. The signals from all the PMTs are then combined to reconstruct the position and energy of the original neutrino interaction. This technique allows JUNO to measure the energy of the incoming neutrino with high precision, which is crucial for studying neutrino oscillation. %When a photon hits the photocathode (the light-sensitive surface inside the PMT), it can eject an electron through the photoelectric effect. This electron is then accelerated by an electric field towards a series of electrodes called dynodes. Each time an electron hits a dynode, more electrons are released. This process is repeated multiple times, resulting in a cascade of electrons and a significant amplification of the original signal. The final electrical signal, which can be easily measured, is proportional to the number of photons that hit the photocathode. 


\subsection{Background}
The design and composition of the scintillator in the JUNO experiment have been optimized to minimize background coming from various radiation sources. Despite these efforts, several types of background signals are inevitably produced inside the detector. For the purpose of the analysis, we focus primarily on the three most significant contributors:

\subsubsection*{Radiogenic Backgrounds}


Radiogenic backgrounds in the JUNO experiment primarily originate from the radioactive decay of isotopes such as $^{238}\mathrm{U}$, $^{232}\mathrm{Th}$, and $^{40}\mathrm{K}$. These isotopes are naturally present in the materials the JUNO detector is made of, including the acrylic used for the detector walls, the metal structure supporting the detector, the PMT glass, the gas during early filling phases, and the surrounding water. They are also found in the surrounding rock. These isotopes undergo radioactive decay, emitting various forms of radiation. The decay of $^{238}\mathrm{U}$ and $^{232}\mathrm{Th}$ occurs through decay chains, where each isotope successively decays into different isotopes, releasing radiation in the process. The emitted radiation includes alpha particles, beta particles, and gamma rays. As for $^{40}\mathrm{K}$, it undergoes beta decay to $^{40}\mathrm{Ca}$ or electron capture to $^{40}\mathrm{Ar}$, resulting in the emission of a gamma ray. These radiogenic backgrounds need to be carefully accounted for and minimized to accurately detect reactor antineutrinos in the JUNO experiment.

The coincidence of two otherwise uncorrelated events, typically of radiogenic origin, forms the so-called \textit{Accidental Background}. This background  dominates  the low  energy  part  of  the  spectrum  due to its nature.

Accidental background can potentially mimic the signal from IBD in several ways:

\begin{enumerate}
	\item \textbf{Beta decays and electron captures}: These processes result in the emission of electrons or positrons, which can produce a scintillation signal similar to the prompt signal from IBD.

	\item \textbf{Gamma rays}: High-energy gamma rays can Compton scatter in the detector, producing electrons with enough energy to mimic the prompt signal from IBD. In addition, gamma rays can produce electron-positron pairs in the detector, which can mimic both the prompt and delayed signals from IBD.
	
	\item \textbf{Neutrons}: Some decays in the $^{238}\mathrm{U}$ and $^{232}\mathrm{Th}$ chains emit neutrons, which can be captured on protons in the detector, mimicking the delayed signal from IBD.

\end{enumerate}

The goal of this thesis work is to study these accidental background events, a topic that will be discussed in detail in the following sections.



The term \textit{Correlated Background} refers to cosmogenic backgrounds and other sources of $\bar{\nu}$, which will be discussed in detail below.

These background sources stem from a singular physics process and generate both a prompt and delayed signal, closely mimicking the IBD events originating from the reactor as listed in Table \ref{tab:IBD_reactor_source}. These events share essential characteristics with IBD events, including a prompt signal, a delayed signal, and a similar temporal separation. As a result, these events are indistinguishable from IBD events, presenting substantial challenges in terms of background reduction through cuts or other more sophisticated techniques.

\subsection*{Cosmogenic Backgrounds}

Cosmogenic backgrounds in JUNO primarily result from the interaction of cosmic rays, particularly high-energy muons ($\mathcal{O}(\text{GeV})$), with the detector materials. These interactions lead to the production of fast neutrons and unstable isotopes through the process of spallation in which a high-energy particle strikes a target atom, causing it to emit smaller particles such as neutrons and unstable isotopes. These muons interact with the detector materials, resulting in the production of isotopes like  $^{9}\mathrm{Li}$, $^{8}\mathrm{He}$ and $^{11}\mathrm{C}$, which are unstable and subsequently decay, contributing to additional background events.

These fast neutrons and unstable isotopes, produced from the interactions of muons with the detector materials, can generate signals that mimic an IBD event. Specifically, there are two distinct signals to consider.

The first signal is generated by an electron. The energy and momentum of this electron can make it appear like a positron, the particle that would be expected in an IBD event. The second is generated by a neutron. This neutron can be captured by a proton in the detector, producing a signal identical to what would be expected from the neutron in an IBD event.


\subsection*{Additional Sources of $\overline{\nu}_e$}

Other sources of antineutrinos also contribute to the background. Those are geoneutrinos, atmospheric neutrinos, and reactor antineutrinos:\\

\textbf{Geoneutrinos} are antineutrinos produced by natural radioactivity within the Earth. Natural radioactivity exists in materials present in the Earth's crust and mantle, such as $^{238}\mathrm{U}$,$^{232}\mathrm{Th}$ and $^{40}\mathrm{K}$. These materials undergo radioactive decays, generating antineutrinos as decay products, that produce IBD signals.\\

\textbf{Atmospheric neutrinos} are generated by interactions of cosmic rays with the Earth's atmosphere. When high-energy cosmic rays collide with the atmosphere, they produce a cascade of particles, including muons and neutrinos. The muons generated in these interactions can decay, producing antineutrinos.\\

\textbf{World reactors} serve as a significant source of antineutrinos. These reactors, utilized for the production of electrical energy through the process of nuclear fission, also emit antineutrinos as a byproduct of this process. With a total of 832 nuclear reactors globally \cite{word_react}, the collective emission of antineutrinos becomes significant.

Among all the radiogenic processes, only one correlated background requires consideration: the $\mathrm{C}$($\alpha$,n)$^{16}\mathrm{O}$, decay that produces an alpha particle (prompt signal) and a neutron that is captured as delayed, exactly like an IBD, occurring within the liquid scintillator.


Here a viasualization sumary of all the bacgrounds contributions:

\begin{figure}[h]
	\centering
	\includegraphics[width=0.7\linewidth]{Images/backgrounds_spectrum}
	\caption{Visible energy spectrum with (grey) and without (black) backgrounds. The predicted backgrounds, which make up around 7$\%$ of the whole sample of IBD candidates and are primarily confined below $\approx$ 3 MeV, are shown in the inset as spectra, from \cite{Sub_osci}.}
	\label{fig:backgroundsspectrum}
\end{figure}


Following the comprehensive discussion of all background events in the JUNO experiment, it becomes clear that due to the significant presence of various types of background events, efforts are being made to study their contribution. Several strategies have been employed to mitigate these background signals. Methods include the use of shielding materials to block external radiation, careful selection and treatment of detector materials to minimize internal radioactivity, and event selection techniques to identify and reject background events.

However, it's important to note that a large portion of the accidental background events are the only ones where significant reduction can be achieved. These are the events that occur randomly and independently, and their reduction requires a different approach compared to correlated backgrounds. The focus of this thesis is precisely on these accidental background events, exploring strategies and techniques to further minimize their impact on the experiment.\\	 

  \chapter{Framework}
\section{Introduction to Machine Learning}

Machine learning is a powerful tool that can be used to identify patterns in complex datasets. In the context of particle physics, machine learning algorithms can be used to detect signals from background noise in large datasets generated by detectors. In particular, for the detection of IBD signals from background, machine learning algorithms can be used to identify patterns in the data that are indicative of an IBD event, and to distinguish these signals from the background noise. Moreover, one advantage of machine learning for particle physics is that it can handle large amounts of data and identify subtle patterns that may be difficult for humans to detect.
\\

There are several machine learning algorithms that can be used for this purpose, including decision trees, deep neural networks and support vector machines. These algorithms can be trained on simulated data to recognize IBD events signals over the background. 


\subsection{Supervised Learning}

In supervised learning, the algorithm is trained on a labeled dataset, where the input data is accompanied by the correct output. The goal of the algorithm is to learn a function that can map input data to output data. Some examples of supervised learning algorithms include linear regression, logistic regression, decision trees, and support vector machines.

To understand the consepts of supervised learning it is useful to discuss a simle machin learning algorithm, linear regression. 

\section{Linear Regression}
Linear regression is a type of supervised learning algorithm used in machine learning for predictive analysis. It is used to model the relationship between a dependent variable (also called the target variable) and one or more independent variables (also called the features or predictors).

The basic idea behind linear regression is to find the best-fitting line that describes the relationship between the independent and dependent variables. This line is often called the regression line or the line of best fit. The equation of this line can be written as:

\begin{equation}
	y = \beta_0 + \beta_1 x_1 + \beta_2 x_2 + ... + \beta_p x_p + \epsilon
\end{equation}

where $y$ is the dependent variable, $x_1, x_2, ..., x_p$ are the independent variables, $\beta_0, \beta_1, \beta_2, ..., \beta_p$ are the coefficients or parameters of the model, and $\epsilon$ is the error term. The error term captures the deviation of the actual values of the dependent variable from the predicted values.

In order to determine the values of the coefficients $\beta_0, \beta_1, \beta_2, ..., \beta_p$, a common approach is to minimize a loss function, which measures the difference between the predicted values of the dependent variable and the actual values. The most commonly used loss function in linear regression is the mean squared error (MSE) function, which is defined as:

\begin{equation}
	L(\beta_0, \beta_1, \beta_2, ..., \beta_p) = \frac{1}{n} \sum_{i=1}^{n} (y_i - \hat{y_i})^2
\end{equation}

where $n$ is the number of observations, $y_i$ is the actual value of the dependent variable for the $i$-th observation, and $\hat{y_i}$ is the predicted value of the dependent variable for the $i$-th observation.

The goal of linear regression is to find the values of the coefficients $\beta_0, \beta_1, \beta_2, ..., \beta_p$ that minimize the loss function $L(\beta_0, \beta_1, \beta_2, ..., \beta_p)$. This can be achieved using various optimization techniques such as gradient descent or normal equations.

However, it is important to note that linear regression can suffer from overfitting or underfitting. Overfitting occurs when the model is too complex and captures noise in the data, while underfitting occurs when the model is too simple and fails to capture the underlying patterns in the data. To prevent overfitting or underfitting, regularization techniques such as Ridge regression or Lasso regression can be used.



  \chapter{Analysis}

\section{Datasets}
The labeled datasets employed in the ensuing analysis are all products of Monte Carlo simulation, generated via the SNiPER software.

The first of the datasets provided is specifically tailored for the study of Inverse Beta Decay (IBD) events. Each event within this dataset, simulated and injected into the system, is tagged with a unique Simulation Identifier, or SimID. Furthermore, events which trigger a sufficient number of PMTs to be captured by the electronic system are assigned an EventID. This intricate labeling system allows for a clear differentiation between correlated IBD events, which represent actual IBD occurrences, and uncorrelated IBD events.


The second dataset focuses primarily on radioactivity events. Similar to the IBD dataset, it encompasses a large number of simulated events, each reflecting the complex reality of real-world physics phenomena. Additionally, the inherent electronic noise prevalent in actual physical environments is accurately accounted for, ensuring a realistic representation within the simulated context.

In this research undertaking, my central task will focus on a detailed examination and evaluation of the provided datasets. My work will primarily involve not just interpreting the inherent characteristics and peculiarities of the recorded events, but also harnessing these insights to construct comprehensive feature tables. These feature tables, generated from the datasets, will serve as the basis for my subsequent analysis and interpretation, a process which will be elaborated upon in the following sections of this study. The aim is to provide a meaningful understanding of the correlations and implications of these events within the broader context of the JUNO experiment.

%TODO-> Inserire il fatto che i dataset sono stati creati in ordine temporale in parallelo con SimID Per simulare 1500000 eventi nel MonteCarlo ci vorrebbero mesi e mesi di tempo macchina.
%Per affrontare il problema, si parallelizza la simulazione su una infrastruttura chiamata DCI (Distributed Computing Infrastructure).
%In questo modo si può ad esempio dividere i 1500000 eventi in 1500 jobs (simulati quindi da 1500 CPU diverse) da 1000 event ciascuno, completando la produzione in poche ore invece che in mesi.
%Questo approccio ha però il drawback che ogni simulazione parallela sarà indipendente dalle altre e quindi per ciascuna di queste il tempo, i SimID e tutte le altre quantità partirano da 0"" 

%La seconda colonna -> Si, sono il numero di eventi generati. Non tutti gli eventi generati interagiscono con il detector: un evento generato nella struttura metallica può venire fermato nella struttura stessa e non interagire con il detector, o ancora un evento generato nell'acrilico può venire emesso verso l'esterno e quindi non entrare mai in contatto con il detector e così via.
%Se consideri ora solo gli eventi che davvero sono entrati in contatto con il detector, non tutti gli eventi vengono ricostruiti: alcuni eventi sono così poco energetici che non sono sufficienti ad illuminare i PMT e quindi ad azionare il trigger per l'acquisizione dell'evento.

%Per darti un'idea, la rate di decadimenti radioattivi è circa 6.6 MHz, ne vengono effettivamente registrati dal detector appena 100 Hz (o poco meno)

%Questa è una delle sfide più grandi, perchè devi portare un segnale di circa 100 Hz (8.6 x 10e6 eventi/giorno) a rappresentare meno di 1 evento/giorno di background
%-------------------

%Nel MonteCarlo sono inclusi tutti gli effetti che ci aspettiamo nella realtà. E purtroppo, nella realtà, l'elettronica ha un rumore intrinseco (ad esempio puoi cercare in letteratura il Dark Count Rate per i PMTs).
%Ora, nel MonteCarlo vengono iniettati degli eventi (ad esempio una coppia prompt-delayed da IBD). A questo evento iniettato viene associato un SimID (simulation identifier) incrementale. Siccome prompt e delayed sono stati iniettati assieme, avranno lo stesso SimID.
%Non tutti gli eventi però sono registrati e salvati dall'elettronica. In JUNO, un evento viene salvato solo se questo è in grado di illuminare nello stesso momento un gran numero di PMTs (corrispondenti a qualche centinaio di keV).
%Se questo avviene, viene generato un trigger, viene salvato l'evento e gli viene associato un EventID incrementale.
%Può succedere che, per puro caso, il rumore intrinseco dell'elettronica "accenda" nello stesso momento un gran numero di PMT. In questo caso, nessun evento vero di fisica ha generato questo trigger, ma l'elettronica acquisisce e salva comunque questo evento, perch`ha superato la soglia di trigger. Quello che succede in questo caso è che l'evento avrà un EventID, ma nessun SimID associato, perchè nessun evento è stato davvero iniettato nella simulazione.
%Al contrario, può succedere che tu inietti una certa particella a cui quindi viene associato un SimID), ma questa non riesce a illuminare abbastanza PMT, l'evento non viene salvato e quindi non gli viene associato nessun EventID e non viene salvato nel file finale.


%Prima coppia simulata si becca SimID di 0. Gli Event IDs associati saranno 0 e 1.
%Seconda coppia generata avrà SimID di 1. EventIDs in questo caso sarnno 2 e 3. Ecc ecc.
%Sono incrementali per convenzione.

%TODO-> Spiega cosa c'è nei datasets -- ['EventID', 'SimID', 'timestamp', 't_sec', 't_nanosec', 'recx', 'recy', 'recz', 'm_QEn', 'm_pe']
%TODO-> Mention unbalanced datasets problem


%Ciao Fabio, la radioattività con cui hai lavorato rappresenta tutto il contributo da decadimenti radioattivi (alfa, beta, gamma) interno ed esterno a detector, ma che hanno depositato energia all'interno del detector.Ti lascio la tabella degli isotopi e delle rate di decadimento generate qui di seguito:

 

\begin{table}[htp]
\centering
\begin{tabular}{|c|c|c|}
	\hline
 \textbf{Dataset Name}  & \textbf{Number of Events}  & \textbf{Rates (used in elecsim)}   \\\hline\hline
      U238@LS      &   1,000,000 events    &           3.234 Hz            \\\hline
     Th232@LS      &   1,000,000 events    &           0.733 Hz            \\\hline
      K40@LS       &   1,000,000 events    &           0.53 Hz             \\\hline
     Pb210@LS      &   1,000,000 events    &           17.04 Hz            \\\hline
      C14@LS       & 1,000,000,000 events  &           3.3e4 Hz            \\\hline
      Kr85@LS      &   1,000,000 events    &           1.163 Hz            \\\hline
   U238@Acrylic    &   10,000,000 events   &           98.41 Hz            \\\hline
   Th232@Acrylic   &   10,000,000 events   &           22.29 Hz            \\\hline
    K40@Acrylic    &   10,000,000 events   &          161.25 Hz            \\\hline
   U238@node/bar   &  100,000,000 events   &          2102.36 Hz           \\\hline
  Th232@node/bar   &  100,000,000 events   &          1428.57 Hz           \\\hline
   K40@node/bar    &  100,000,000 events   &           344.5 Hz            \\\hline
   Co60@node/bar   &  100,000,000 events   &           97.5 Hz             \\\hline
   U238@PMTGlass   & 1,000,000,000 events  &          4.90e6 Hz            \\\hline
  Th232@PMTGlass   & 1,000,000,000 events  &          8.64e5 Hz            \\\hline
   K40@PMTGlass    & 1,000,000,000 events  &          4.44e5 Hz            \\\hline
  Tl208@PMTGlass   & 1,000,000,000 events  &          1.39e5 Hz            \\\hline
    Co60@Truss     &           0           &             ? Hz              \\\hline
    Tl208@Truss    &           0           &             ? Hz              \\\hline
 Rn222@WaterRadon  &  100,000,000 events   &            90 Hz              \\\hline
\end{tabular}
\caption{Here Caption}
\label{tab:BKG_gen}
\end{table}

\newpage

\section{Feature creation}
The development of machine learning models for the detection of Inverse Beta Decay (IBD) events necessitates a systematic and efficient approach to feature engineering. This process begins with the loading of two separate datasets, one for IBDs and one for radioactivity background, each containing a multitude of potential IBD events. The primary objective is to construct a feature table that encapsulates the unique characteristics of these events, providing a robust foundation for subsequent model training.

\subsection{IBD dataset}
As we mentioned earlier, an IBD event is characterized by two distinct signals with different energies, positions, and times. The first, known as the prompt signal, is caused by the annihilation of a positron with an electron in the scintillator liquid. This interaction yields a signal with a characteristic energy. The second, the delayed signal, results from the capture of a neutron by the scintillator liquid. This signal occurs with a significant delay, at a different position, and with a different energy compared to the prompt signal.

To create the feature table, all possible pairs of events within the dataset were considered, without repetition. Each possible combination was ordered temporally, meaning the second event followed the first. This temporal ordering is crucial in feature determination. Given a pair $i-j$, and considering that neutron capture occurs temporally subsequent to electron-positron annihilation, the following features were constructed:


\begin{itemize}
	 
	\item \textbf{$R_{prompt}$}: This feature represents the distance of the prompt signal, calculated as the distance from the origin to the point $(x_i, y_i, z_i)$ in the detector space where the prompt signal occurred.	
	
	\item $R_{delayed}$: Similar to $R_{prompt}$, this feature represents the distance of the delayed signal, calculated as the distance from the origin to the point $(x_j, y_j, z_j)$ in the detector space where the delayed signal occurred.

	\item \textbf{$E_{prompt}$}: This feature represents the energy of the prompt signal. It captures the characteristic energy released during the annihilation of a positron with an electron in the scintillator liquid.

	\item \textbf{$E_{delayed}$}: This feature represents the energy of the delayed signal. It captures the energy released when a neutron is captured by the scintillator liquid. This capture can occur by hydrogen, resulting in a gamma ray with an energy of approximately 2.2 MeV, or by carbon, resulting in gamma rays with combined energies of about 4.95 MeV to 5.12 MeV.

	\item \textbf{$\Delta_{Time}$}: This feature represents the time difference between the two events. It captures the temporal delay between the occurrence of the prompt and delayed signals.

	\item \textbf{$\Delta_{Radius}$}: This feature represents the spatial distance between the two events. It captures the spatial separation between the points in the detector space where the prompt and delayed signals occurred.

\end{itemize}
These features encapsulate the temporal and spatial differences between the prompt and delayed signals, as well as their respective energies, providing a comprehensive representation of the unique characteristics of IBD events.


\subsubsection*{Event Labeling}

In the context of supervised learning, the process of labeling is crucial as it provides the ground truth against which the performance of the machine learning model is evaluated. In this scenario, each pair of events in the dataset is assigned a label that indicates whether it represents a true Inverse Beta Decay (IBD) event or a background signal (BKG).

The label is a binary value: a label of 1 signifies a true IBD event, while a label of 0 signifies a BKG event. The assignment of these labels is not arbitrary but is guided by a specific criterion based on the simulation identifier (SimID) associated with each event pair.

The SimID is a unique identifier assigned to each simulated event pair during the generation of the dataset. If a pair of events share the same SimID, it means they were generated as part of the same simulation and thus are considered to represent a true IBD event. In this case, they are assigned a label of 1.

Conversely, if a pair of events do not share the same SimID, it means they were generated as part of different simulations. These events are not correlated and thus are considered to represent BKG events. In this case, they are assigned a label of 0.

This labeling strategy based on the SimID ensures a systematic and consistent methodology for event classification. It provides a clear and objective criterion to distinguish between true IBD events and BKG events, which is essential for the training and evaluation of the machine learning model.


\subsubsection*{Efficient Feature Calculation}
Given the large size of the dataset and the computational complexity of feature calculation, a parallel computing approach was adopted to enhance efficiency. The feature calculation task was divided into multiple sub-tasks that could be executed simultaneously by different cores of a CPU. This parallelization significantly reduced the total computation time, particularly beneficial when working with large volumes of data.


To further optimize the computation, a method was implemented to only consider event pairs where the delayed event occurs within a time window of $5*\tau$ from the prompt event. This approach is based on the fact that the time delay between the prompt and delayed events in Inverse Beta Decay (IBD) typically follows an exponential distribution, a characteristic of radioactive decay processes. While this method significantly reduces the number of potential event pairs, it might exclude about $0.7\%$ of IBD events that occur outside this time window. 

While this percentage is relatively small, it's important to consider the potential impact on the analysis results.

\subsection{Radioactivity dataset}
For the radioactivity dataset, the feature calculation was performed in a manner analogous to the IBD dataset. The key difference is that event pairs from the radioactivity dataset are labeled as BKG events, hence assigned a label of 0.



\begin{figure}[h]
	\centering
	\includegraphics[width=1\linewidth]{Images/hist_features.png}
	\caption{Features histograms}
	\label{fig:hist_features}
\end{figure}



In summary, the feature engineering process for IBD event detection involves careful consideration of the unique characteristics of these events, systematic feature construction, and efficient computation strategies. This process provides a robust foundation for the development and training of machine learning models for IBD event detection.

\section{Models}


In the context of the JUNO experiment, a significant part of the effort involves the implementation and optimization of an event selection algorithm. The primary objective of this algorithm is to identify and select Inverse Beta Decay events induced by reactor antineutrinos, which are of paramount importance in the study of neutrino oscillations.

In this chapter, we will present several algorithms for event selection. The first algorithm is a manual cut-based approach, \textbf{Manual Cut}, where specific cuts are defined to select events of interest. This approach involves setting criteria based on the physical characteristics of the events and known background noise sources in the detector. The manual cut algorithm allows for precise control over the selection process and enhances the signal-to-background ratio.

In addition to the manual cut algorithm, other algorithms discussed in this chapter are based on machine learning models, specificaly based on \textbf{Boosted Decision Trees} and on \textbf{Neural Network}. 

The inclusion of machine learning algorithms in event selection provides an alternative approach that can adapt to complex and evolving data patterns. These algorithms can uncover subtle features in the data that may not be easily captured by manual cuts alone.

By exploring both manual cut and machine learning-based algorithms, we aim to provide a comprehensive understanding of different approaches to event selection, highlighting their strengths and limitations in the context of the JUNO experiment.

\subsection{Manual Cut}
The algorithm is designed to suppress various types of background noise while maintaining high efficiency for true IBD events. The selection criteria, or "cuts", are implemented using Python Code,and are applied to the Features Tables discussed above. Each cut within the algorithm serves a distinct purpose in the overall event selection process.

The key components of the event selection algorithm are as follows:

\begin{enumerate}
	\item \textbf{Delta Time ($\Delta t$) and Delta Radius ($\Delta R$) cuts}: The first cut is applied on the time delay and the radial distance between the prompt and delayed signals. The criteria are:
	\begin{itemize}
		\item Time separation between the prompt and delayed signals should be less than 1.0 ms.
		\item Spatial 3D separation should be less than 1.5 m.
	\end{itemize}

	These cuts are designed to reduce accidental background noise, which is the coincidence of two otherwise uncorrelated events, typically of radiogenic origin. The accidental background can be measured with excellent precision and subtracted by off-time window techniques. By imposing a limit on the time and spatial separation between the prompt and delayed signals, the algorithm can effectively distinguish between true IBD events and accidental coincidences.
	\item \textbf{Energy of the Prompt Signal ($E_{pro}$) Cut}: The next cut is applied on the energy of the prompt signal, which is the initial signal produced by the antineutrino interaction. The criteria are:
	\begin{itemize}
		\item Energy of the prompt signal should be within the [0.7, 12.0] MeV range.
	\end{itemize}
	
	This cut is based on the expectation that IBD events dominate this energy range. The energy of the prompt signal corresponds to the energy of the positron from the IBD reaction, and the specific range is chosen to maximize the signal-to-background ratio.
	\item \textbf{Energy of the Delayed Signal ($E_{del}$) Cut}: The final cut is applied on the energy of the delayed signal, which is the signal produced by the neutron capture that follows the antineutrino interaction. The criteria are:
	\begin{itemize}
		\item Energy of the delayed signal should be within the [1.9, 2.5] MeV or [4.4, 5.5] MeV ranges.
	\end{itemize}
	
	These energy selection windows correspond to neutron capture on hydrogen and carbon, respectively. The energy of the delayed signal is characteristic of the neutron capture process, and the specific ranges are chosen to correspond to the expected energies of neutron capture on hydrogen and carbon in the detector.
\end{enumerate}

An evaluation was conducted to assess the accuracy of the algorithm. In this evaluation, the feature table of true IBD events was utilized to gauge the algorithm's performance. The results indicated that out of a total of 1,468,385 events, 1,435,115 events were accurately classified as IBD. This achievement translates to an accuracy rate of $97.73\%$, indicating the algorithm's successful identification of the majority of true IBD events.\\


Furthermore, the efficiency calculation was performed on the background dataset, yielding the following results: 26.0 out of 993,457 events were mistakenly classified as IBD, resulting in an efficiency rate of $99.997\%$. This indicates that the algorithm has an excellent capability to distinguish and reject background events, achieving a high level of efficiency in identifying true IBD events. The extremely low misclassification rate of background events as IBD further highlights the algorithm's effectiveness in minimizing false positives and maintaining a high level of purity in the selected IBD events.


A summary table presents the results obtained from the evaluations:

\begin{table}[ht]
	\centering
	\caption{Summary of Algorithm Performance}
	\label{tab:algorithm-results}
	\begin{tabular}{|c|c|c|c|}
		\hline
		Dataset & Total Events & IBD Events Selected & Accuracy (\%) \\ \hline\hline
		True IBD Events & 1,468,385 & 1,435,115 & 97.73 \\ \hline
		Background Events & 993,457 & 26 & 99.997 \\ \hline
	\end{tabular}
\end{table}

\subsection{XGBoost}
\subsection{PyThorch}

\section{Conclusion}

% Please add the following required packages to your document preamble:
% \usepackage[table,xcdraw]{xcolor}
% If you use beamer only pass "xcolor=table" option, i.e. \documentclass[xcolor=table]{beamer}
\begin{table}[h!]
	\begin{tabular}{lllll}
		\cline{1-3}
		\multicolumn{1}{|l|}{} & \multicolumn{1}{c|}{{\color[HTML]{CE6301} \textbf{Manual Cut Algorithm}}} & \multicolumn{1}{l|}{{\color[HTML]{009901} \textbf{BDT Algorithm}}} &  &  \\ \cline{1-3}
		\multicolumn{1}{|l|}{\textit{Radioactivity}} & \multicolumn{1}{l|}{\begin{tabular}[c]{@{}l@{}}Efficiency: 99.9973\%\\ Purity: 100\%\end{tabular}} & \multicolumn{1}{l|}{\begin{tabular}[c]{@{}l@{}}Efficiency: 99.997684\%\\ Purity: 100\%\end{tabular}} &  &  \\ \cline{1-3}
		\multicolumn{1}{|l|}{\textit{True IBDs}} & \multicolumn{1}{l|}{\begin{tabular}[c]{@{}l@{}}Efficiency: 97.734\%\\ Purity:100\%\end{tabular}} & \multicolumn{1}{l|}{\begin{tabular}[c]{@{}l@{}}Efficiency: 99.997616\%\\ Purity: 100\%\end{tabular}} &  &  \\ \cline{1-3}
		&  &  &  & 
	\end{tabular}
\end{table}


  \chapter{Conclusions}
In this thesis, we have conducted an investigation of models for event classification in the JUNO experiment. Our goal was to compare the performance of different models and evaluate their effectiveness in distinguishing between Inverse Beta Decay events and background events.

The Manual Cut model, XGBoost model, and PyTorch Neural Network were explored as potential solutions for event classification.

Based on the evaluation of overall accuracy and purity, the XGBoost model emerges as the optimal choice for the identification of both Inverse Beta Decay and background  events. It consistently demonstrates high accuracy and purity in distinguishing between the two event types. The PyTorch Neural Network model also exhibits strong accuracy in recognizing IBD events, although its purity is slightly lower compared to XGBoost. On the other hand, the Manual Cut model, while having high background efficiency, shows lower overall accuracy when compared to the machine learning models.

However, there are still several areas that deserve further exploration and hold potential for future advancements:
\begin{itemize}
    \item Feature Enhancement: Incorporating additional features, such as reconstructed energy, vertex position, and spatial distribution of hits, could potentially improve the performance of the models and enhance event classification accuracy.
    \item Model Optimization: Fine-tuning the hyperparameters of the machine learning models and exploring different architectures could further enhance their performance and robustness.
\end{itemize}
  
%	\input{section/intro}
% 	\input{section/more}
%   \input{section/trials}
%   \input{section/results}
%   \input{section/conclusions}

%%%%%%%%%%%%%%%%%%%%%%%%%%%%%%%%%%%%%% BIBLIOGRAPHY %%%%%%%%%%%%%%%%%%%%%%%%%%%%%%%%%%%%%%

  \backmatter


  \printbibliography





\end{document}
