\chapter{Conclusions}
In this thesis, we have conducted an investigation of models for event classification in the JUNO experiment. Our goal was to compare the performance of different models and evaluate their effectiveness in distinguishing between Inverse Beta Decay events and background events.

The Manual Cut model, XGBoost model, and PyTorch Neural Network were explored as potential solutions for event classification.

Based on the evaluation of overall accuracy and purity, the XGBoost model emerges as the optimal choice for the identification of both Inverse Beta Decay and background  events. It consistently demonstrates high accuracy and purity in distinguishing between the two event types. The PyTorch Neural Network model also exhibits strong accuracy in recognizing IBD events, although its purity is slightly lower compared to XGBoost. On the other hand, the Manual Cut model, while having high background efficiency, shows lower overall accuracy when compared to the machine learning models.

However, there are still several areas that deserve further exploration and hold potential for future advancements:
\begin{itemize}
    \item Feature Enhancement: Incorporating additional features, such as reconstructed energy, vertex position, and spatial distribution of hits, could potentially improve the performance of the models and enhance event classification accuracy.
    \item Model Optimization: Fine-tuning the hyperparameters of the machine learning models and exploring different architectures could further enhance their performance and robustness.
\end{itemize}